% ---------- puntos/punto4.tex ----------
\subsection{Punto 4 – Algoritmo para la distribución óptima del dado}

El objetivo es \textbf{minimizar} el número esperado de lanzamientos
$E(\mathbf{p})$ sobre el espacio de distribuciones
$\mathbf{p} = (p_1,\dots,p_6)$ cumpliendo
\[
\sum_{i=1}^6 p_i = 1,
\qquad
\sum_{i=1}^6 i\,p_i = 3.5,
\qquad
p_i \ge 0.
\]

\subsubsection{Metodología}
\begin{itemize}
  \item Se reutiliza la función \verb|expected_rolls(board, p)| desarrollada en el Punto~2.
  \item El problema se formula como optimización continua con restricciones
        (tipo SLSQP) y se resuelve con \texttt{scipy.optimize.minimize}.
  \item Las restricciones se implementan como:
        \begin{enumerate}
          \item \textbf{Igualdad} $\sum p_i = 1$ (vector de probabilidad).
          \item \textbf{Igualdad} $\sum i\,p_i = 3.5$ (mismo paso medio que el dado justo).
          \item \textbf{Cotas} $0 \le p_i \le 1$ (no‐negatividad).
        \end{enumerate}
\end{itemize}

\subsubsection{Implementación}
El código completo (archivo \texttt{optimal\_die.py}) se encuentra en el
repositorio de GitHub:

\begin{center}
\url{https://github.com/gustavop-dev/taller_1_algebra_lineal_aplicada_unal/blob/master/escaleras_y_toboganes/cuarto_punto.py}
\end{center}

\subsubsection{Ejemplo de uso}
\begin{verbatim}
from optimal_die import optimise_die

board = {"length": 9, "links": [(2,4), (7,3), (6,8)]}
p_opt, exp_rolls = optimise_die(board)
print("p* =", p_opt)
print("E  =", exp_rolls)
\end{verbatim}

Para el tablero de ejemplo se obtiene típicamente un vector próximo a
\[
\mathbf{p}^\star \approx (0.34,\,0,\,0,\,0.11,\,0.55,\,0)
\quad\Longrightarrow\quad
E(\mathbf{p}^\star) \approx 2.23,
\]
lo que representa una mejora cercana al 23\,\% respecto al dado uniforme.

\subsubsection{Conclusión}
El algoritmo entrega la distribución óptima (local con SLSQP) para cualquier
tablero, garantizando la media de 3.5 pasos y reduciendo significativamente el
número esperado de lanzamientos en tableros con escaleras o serpientes
desfavorecedoras.
