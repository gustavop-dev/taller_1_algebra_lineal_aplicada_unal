\subsection{Dimensión del espacio conmutante}

Sea $A\in\mathbb{R}^{n\times n}$ (o $\mathbb{C}^{n\times n}$). Denotemos por
\[
\mathcal{C}(A)=\{\,X\in\mathbb{R}^{n\times n}:\; XA=AX\,\}
\]
el \emph{conmutante} de $A$.  Su dimensión está gobernada por la
estructura de Jordan (o, si $A$ es diagonalizable, por las multiplicidades
algebraicas de sus valores propios).

\subsubsection{Fórmula general}
Sea $\sigma(A)=\{\lambda_1,\dots,\lambda_r\}$ el conjunto de valores propios
distintos y sea $m_k$ la multiplicidad algebraica de $\lambda_k$. Entonces
\[
\dim \mathcal{C}(A)=\sum_{k=1}^{r} m_k^{\,2}.
\]
(Referencia: Hoffman–Kunze, \emph{Linear Algebra}, cap.~4.)

\subsubsection{Dimensión mínima}
\begin{itemize}
  \item \textbf{Valor}: $\min \dim\mathcal{C}(A)=n$.
  \item \textbf{Condición}: $A$ tiene $n$ valores propios \emph{distintos}
        (espectro simple).  Entonces $m_k=1$ y la fórmula da
        $\sum 1^{2}=n$.  Equivalente a que $A$ sea diagonalizable con
        multiplicidad unitaria.
\end{itemize}

\subsubsection{Dimensión máxima}
\begin{itemize}
  \item \textbf{Valor}: $\max \dim\mathcal{C}(A)=n^{2}$.
  \item \textbf{Condición}: $A$ es un múltiplo escalar de la identidad,
        $A=\lambda I$. Entonces $XA=AX$ para toda matriz $X$, de modo que
        $\mathcal{C}(A)=\mathbb{R}^{n\times n}$.
\end{itemize}

\subsubsection{Valores intermedios}
Para espectros con repeticiones parciales ($1<m_k<n$) la dimensión toma
valores intermedios conforme a la fórmula general.  Por ejemplo,
$A=\operatorname{diag}(1,1,2)$ tiene multiplicidades $m_1=2$, $m_2=1$ y
$\dim\mathcal{C}(A)=2^{2}+1^{2}=5$.