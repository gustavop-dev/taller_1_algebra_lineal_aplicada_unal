\subsection{Punto 3 – Distribuciones con media $3.5$ más convenientes que la uniforme}

\subsubsection{Planteamiento}
Sea $E(\mathbf{p})$ el número esperado de tiradas al emplear una distribución de dado
$\mathbf{p} = (p_1,\dots,p_6)$.  Buscamos un $\mathbf{p}$ que satisfaga
\begin{align}
\sum_{i=1}^6 p_i &= 1,
&
\sum_{i=1}^6 i\,p_i &= 3.5, \label{eq:moment}
\end{align}
pero tal que $E(\mathbf{p}) < E(\mathbf{u})$, donde
$\mathbf{u}=(\tfrac16,\dots,\tfrac16)$ es la distribución uniforme.

\subsubsection{Observaciones sobre el tablero}
\begin{itemize}
  \item \textbf{Casillas ventajosas}: caer en \(2\) o \(6\) activa una escalera que adelanta
        a \(4\) y \(8\), respectivamente.
  \item \textbf{Casilla perjudicial}: caer en \(7\) activa una serpiente que retrocede a \(3\).
  \item Por tanto, conviene aumentar la probabilidad de lanzamientos que conduzcan a \(2\) y \(6\), 
        y reducir la de resultados que terminen en \(7\).
\end{itemize}

\subsubsection{Ejemplo de distribución mejorada}
Una búsqueda exhaustiva (discretizando $p_i$ en pasos de $0.05$)
produjo la distribución
\begin{equation}\label{eq:best}
\mathbf{p}^\star = (0.35,\,0,\,0,\,0.10,\,0.55,\,0),
\end{equation}
que cumple las restricciones \eqref{eq:moment} concentrando la masa
en las caras \(1\), \(4\) y \(5\).
El valor esperado sigue siendo \(3.5\):
\[
1\cdot 0.35 \;+\; 4\cdot 0.10 \;+\; 5\cdot 0.55 \;=\; 3.5.
\]

\subsubsection{Comparación de desempeño}
La Tabla~\ref{tab:results} muestra el número esperado de lanzamientos obtenido
con el dado uniforme y con $\mathbf{p}^\star$ (cálculos mediante el algoritmo del Punto 2).

\begin{table}[h]
  \centering
  \caption{Esperanza de lanzamientos en el tablero de ejemplo}
  \label{tab:results}
  \begin{tabular}{lcc}
    \toprule
    Distribución & $E(\mathbf{p})$ & Mejora \\\midrule
    Uniforme $\mathbf{u}$ & 2.9072 & --- \\
    $\mathbf{p}^\star$ (ec.~\ref{eq:best}) & 2.2354 & $\approx$\,23 \% menos \\
    \bottomrule
  \end{tabular}
\end{table}

\subsubsection{Interpretación}
\begin{itemize}
  \item \textbf{Alta probabilidad en 1}: desde la casilla inicial, un \emph{1} lleva a \(2\)
        y asciende a \(4\), avanzando dos casillas netas.
  \item \textbf{Alta probabilidad en 5}: un \emph{5} conduce a \(6\) y luego a \(8\),
        quedando a un paso de la meta.
  \item \textbf{Probabilidad nula en 6}: así se evita caer en la serpiente de la casilla \(7\)
        en el primer turno.
\end{itemize}

\subsubsection{Conclusión}
Existen distribuciones no uniformes con la misma media que un dado regular
que disminuyen el número esperado de lanzamientos en el tablero propuesto.
El vector \eqref{eq:best} es un ejemplo concreto, reduciendo la expectativa
en aproximadamente un 23 \%.
