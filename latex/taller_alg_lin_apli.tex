% !TeX encoding = UTF-8
% =====================================================
% Archivo principal: main.tex
% Taller N°1 – Álgebra Lineal Aplicada
% =====================================================
\documentclass{article}

% -----------------------------------------------------
% Paquetes básicos y configuración
% -----------------------------------------------------
\usepackage[utf8]{inputenc}
\usepackage[T1]{fontenc}
\usepackage{lmodern}
\usepackage[spanish]{babel}
\usepackage{amsmath, amssymb, amsfonts}
\usepackage{booktabs}
\usepackage{hyperref}
\usepackage{geometry}
\geometry{margin=1in}

% -----------------------------------------------------
% Datos del autor y título del taller
% -----------------------------------------------------
\title{Taller N°1 \\ \textit{Álgebra Lineal Aplicada}}
\author{Gustavo Adolfo P\'erez P\'erez\\Universidad Nacional de Colombia -- Sede Medell\'in\\Programa de Ciencias de la Computaci\'on}
\date{\today}

\begin{document}
\maketitle

% =====================================================
% Primera parte: Escaleras y Toboganes
% =====================================================
\section{Escaleras y Toboganes}
\subsection{Punto 1 – Algoritmo para dado justo}
 El código desarrollado resuelve el sistema de ecuaciones de la cadena de
 Markov inducida por el tablero y un dado regular (probabilidad $1/6$ en
 cada cara).
 
\vspace{0.3em}
\noindent\emph{Repositorio:} \url{https://github.com/gustavop-dev/taller_1_algebra_lineal_aplicada_unal/blob/master/escaleras_y_toboganes/primer_punto.py}
\subsection{Punto 2 – Probabilidades a largo plazo}
El programa calcula las probabilidades estacionarias del juego.

\vspace{0.3em}
\noindent\emph{Repositorio:} \url{https://github.com/gustavop-dev/taller_1_algebra_lineal_aplicada_unal/blob/master/escaleras_y_toboganes/segundo_punto.py}

\subsection{Punto 3 – Distribuciones con media $3.5$ más convenientes que la uniforme}

\subsubsection{Planteamiento}
Sea $E(\mathbf{p})$ el número esperado de tiradas al emplear una distribución de dado
$\mathbf{p} = (p_1,\dots,p_6)$.  Buscamos un $\mathbf{p}$ que satisfaga
\begin{align}
\sum_{i=1}^6 p_i &= 1,
&
\sum_{i=1}^6 i\,p_i &= 3.5, \label{eq:moment}
\end{align}
pero tal que $E(\mathbf{p}) < E(\mathbf{u})$, donde
$\mathbf{u}=(\tfrac16,\dots,\tfrac16)$ es la distribución uniforme.

\subsubsection{Observaciones sobre el tablero}
\begin{itemize}
  \item \textbf{Casillas ventajosas}: caer en \(2\) o \(6\) activa una escalera que adelanta
        a \(4\) y \(8\), respectivamente.
  \item \textbf{Casilla perjudicial}: caer en \(7\) activa una serpiente que retrocede a \(3\).
  \item Por tanto, conviene aumentar la probabilidad de lanzamientos que conduzcan a \(2\) y \(6\), 
        y reducir la de resultados que terminen en \(7\).
\end{itemize}

\subsubsection{Ejemplo de distribución mejorada}
Una búsqueda exhaustiva (discretizando $p_i$ en pasos de $0.05$)
produjo la distribución
\begin{equation}\label{eq:best}
\mathbf{p}^\star = (0.35,\,0,\,0,\,0.10,\,0.55,\,0),
\end{equation}
que cumple las restricciones \eqref{eq:moment} concentrando la masa
en las caras \(1\), \(4\) y \(5\).
El valor esperado sigue siendo \(3.5\):
\[
1\cdot 0.35 \;+\; 4\cdot 0.10 \;+\; 5\cdot 0.55 \;=\; 3.5.
\]

\subsubsection{Comparación de desempeño}
La Tabla~\ref{tab:results} muestra el número esperado de lanzamientos obtenido
con el dado uniforme y con $\mathbf{p}^\star$ (cálculos mediante el algoritmo del Punto 2).

\begin{table}[h]
  \centering
  \caption{Esperanza de lanzamientos en el tablero de ejemplo}
  \label{tab:results}
  \begin{tabular}{lcc}
    \toprule
    Distribución & $E(\mathbf{p})$ & Mejora \\\midrule
    Uniforme $\mathbf{u}$ & 2.9072 & --- \\
    $\mathbf{p}^\star$ (ec.~\ref{eq:best}) & 2.2354 & $\approx$\,23 \% menos \\
    \bottomrule
  \end{tabular}
\end{table}

\subsubsection{Interpretación}
\begin{itemize}
  \item \textbf{Alta probabilidad en 1}: desde la casilla inicial, un \emph{1} lleva a \(2\)
        y asciende a \(4\), avanzando dos casillas netas.
  \item \textbf{Alta probabilidad en 5}: un \emph{5} conduce a \(6\) y luego a \(8\),
        quedando a un paso de la meta.
  \item \textbf{Probabilidad nula en 6}: así se evita caer en la serpiente de la casilla \(7\)
        en el primer turno.
\end{itemize}

\subsubsection{Conclusión}
Existen distribuciones no uniformes con la misma media que un dado regular
que disminuyen el número esperado de lanzamientos en el tablero propuesto.
El vector \eqref{eq:best} es un ejemplo concreto, reduciendo la expectativa
en aproximadamente un 23 \%.

% ---------- puntos/punto4.tex ----------
\subsection{Punto 4 – Algoritmo para la distribución óptima del dado}

El objetivo es \textbf{minimizar} el número esperado de lanzamientos
$E(\mathbf{p})$ sobre el espacio de distribuciones
$\mathbf{p} = (p_1,\dots,p_6)$ cumpliendo
\[
\sum_{i=1}^6 p_i = 1,
\qquad
\sum_{i=1}^6 i\,p_i = 3.5,
\qquad
p_i \ge 0.
\]

\subsubsection{Metodología}
\begin{itemize}
  \item Se reutiliza la función \verb|expected_rolls(board, p)| desarrollada en el Punto~2.
  \item El problema se formula como optimización continua con restricciones
        (tipo SLSQP) y se resuelve con \texttt{scipy.optimize.minimize}.
  \item Las restricciones se implementan como:
        \begin{enumerate}
          \item \textbf{Igualdad} $\sum p_i = 1$ (vector de probabilidad).
          \item \textbf{Igualdad} $\sum i\,p_i = 3.5$ (mismo paso medio que el dado justo).
          \item \textbf{Cotas} $0 \le p_i \le 1$ (no‐negatividad).
        \end{enumerate}
\end{itemize}

\subsubsection{Implementación}
El código completo (archivo \texttt{optimal\_die.py}) se encuentra en el
repositorio de GitHub:

\begin{center}
\url{https://github.com/gustavop-dev/taller_1_algebra_lineal_aplicada_unal/blob/master/escaleras_y_toboganes/cuarto_punto.py}
\end{center}

\subsubsection{Ejemplo de uso}
\begin{verbatim}
from optimal_die import optimise_die

board = {"length": 9, "links": [(2,4), (7,3), (6,8)]}
p_opt, exp_rolls = optimise_die(board)
print("p* =", p_opt)
print("E  =", exp_rolls)
\end{verbatim}

Para el tablero de ejemplo se obtiene típicamente un vector próximo a
\[
\mathbf{p}^\star \approx (0.34,\,0,\,0,\,0.11,\,0.55,\,0)
\quad\Longrightarrow\quad
E(\mathbf{p}^\star) \approx 2.23,
\]
lo que representa una mejora cercana al 23\,\% respecto al dado uniforme.

\subsubsection{Conclusión}
El algoritmo entrega la distribución óptima (local con SLSQP) para cualquier
tablero, garantizando la media de 3.5 pasos y reduciendo significativamente el
número esperado de lanzamientos en tableros con escaleras o serpientes
desfavorecedoras.


% =====================================================
% Segunda parte: Conmutatividad
% =====================================================
\section{Conmutatividad}
El código completo del algoritmo para calcular una base ortonormal del espacio conmutante (\texttt{punto\_inicial.py}) se encuentra en el repositorio de GitHub:

\begin{center}
\url{<RELLENAR_LINK_INICIAL>}
\end{center}

\subsection{Dimensión del espacio conmutante}

Sea $A\in\mathbb{R}^{n\times n}$ (o $\mathbb{C}^{n\times n}$). Denotemos por
\[
\mathcal{C}(A)=\{\,X\in\mathbb{R}^{n\times n}:\; XA=AX\,\}
\]
el \emph{conmutante} de $A$.  Su dimensión está gobernada por la
estructura de Jordan (o, si $A$ es diagonalizable, por las multiplicidades
algebraicas de sus valores propios).

\subsubsection{Fórmula general}
Sea $\sigma(A)=\{\lambda_1,\dots,\lambda_r\}$ el conjunto de valores propios
distintos y sea $m_k$ la multiplicidad algebraica de $\lambda_k$. Entonces
\[
\dim \mathcal{C}(A)=\sum_{k=1}^{r} m_k^{\,2}.
\]
(Referencia: Hoffman–Kunze, \emph{Linear Algebra}, cap.~4.)

\subsubsection{Dimensión mínima}
\begin{itemize}
  \item \textbf{Valor}: $\min \dim\mathcal{C}(A)=n$.
  \item \textbf{Condición}: $A$ tiene $n$ valores propios \emph{distintos}
        (espectro simple).  Entonces $m_k=1$ y la fórmula da
        $\sum 1^{2}=n$.  Equivalente a que $A$ sea diagonalizable con
        multiplicidad unitaria.
\end{itemize}

\subsubsection{Dimensión máxima}
\begin{itemize}
  \item \textbf{Valor}: $\max \dim\mathcal{C}(A)=n^{2}$.
  \item \textbf{Condición}: $A$ es un múltiplo escalar de la identidad,
        $A=\lambda I$. Entonces $XA=AX$ para toda matriz $X$, de modo que
        $\mathcal{C}(A)=\mathbb{R}^{n\times n}$.
\end{itemize}

\subsubsection{Valores intermedios}
Para espectros con repeticiones parciales ($1<m_k<n$) la dimensión toma
valores intermedios conforme a la fórmula general.  Por ejemplo,
$A=\operatorname{diag}(1,1,2)$ tiene multiplicidades $m_1=2$, $m_2=1$ y
$\dim\mathcal{C}(A)=2^{2}+1^{2}=5$.
\subsection{Problema de mínimos cuadrados $\displaystyle\arg\min_{X}\|AX-XA-I\|_{\mathrm{F}}$}

Sea $A \in \mathbb{R}^{n\times n}$ (o $\mathbb{C}^{n\times n}$) fija.
Buscamos la matriz $X$ que minimice
\[
F(X)=\|AX - XA - I\|_{\mathrm{F}}^{2},
\]
donde $\|\cdot\|_{\mathrm{F}}$ es la norma de Frobenius.

\subsubsection{Proyección ortogonal sobre el conmutante}
Recordemos del \textbf{Punto 5} que
\[
\mathcal{C}(A)=\{X \colon XA = AX\}
\]
tiene una base ortonormal (producto Frobenius)  
$\{C_1,\dots,C_d\}$, obtenible con el algoritmo del \textbf{Punto 4}
(archivo \texttt{commuting\_basis.py}).  
Sea $\Pi\colon\mathbb{R}^{n\times n}\!\to\!\mathcal{C}(A)$ la proyección
ortogonal,
\[
\Pi(B)=\sum_{i=1}^{d}\langle B, C_i\rangle\,C_i,
\quad
\langle B, C_i\rangle=\operatorname{trace}(C_i^{\mathsf T}B).
\]

\paragraph{Resultado.}
$F(X)$ es estrictamente convexa; su minimizador único es
\[
X^{\star}=\Pi(I),
\]
la proyección de la identidad sobre $\mathcal{C}(A)$.  
El error mínimo es
\[
F(X^{\star})=\bigl\|I-\Pi(I)\bigr\|_{\mathrm{F}}^{2}.
\]

\subsubsection{Algoritmo práctico}
\begin{enumerate}
  \item Calcular $\{C_i\}$ con \verb|commuting_basis(A)|.
  \item Coeficientes $\alpha_i=\langle I,C_i\rangle$.
  \item $X^{\star}=\sum \alpha_i C_i$.
  \item (Opcional) comprobar $\|AX^{\star}-X^{\star}A-I\|_{\mathrm{F}}$.
\end{enumerate}

\subsubsection{Implementación en Python}
\begin{verbatim}
from commuting_basis import commuting_basis
import numpy as np

def least_squares_commuting(A):
    C = commuting_basis(A)           # ortho basis of commutant
    I = np.eye(A.shape[0])
    X = sum(np.trace(Ci.T @ I) * Ci for Ci in C)  # projection of I
    err = np.linalg.norm(A @ X - X @ A - I, 'fro')
    return X, err
\end{verbatim}

\subsubsection{Observaciones}
\begin{itemize}
  \item Si $A=\lambda I$, entonces $X^{\star}=I$ y el error es $0$.
  \item Si el espectro de $A$ es simple, igualmente $X^{\star}=I$.
  \item Para multiplicidades intermedias, $X^{\star}$ combina los
        proyectores asociados a cada subespacio propio de $A$.
\end{itemize}

\subsection{Tiempo de ejecución del algoritmo \texttt{commuting\_basis}}

El objetivo es estimar empíricamente cómo crece el tiempo de cómputo de
\texttt{commuting\_basis(A)} cuando el tamaño $n$ de la matriz $A$ aumenta.

\subsubsection{Diseño de la simulación}
\begin{itemize}
  \item Se generan matrices aleatorias $A\in\mathbb{R}^{n\times n}$ con
        entradas i.i.d. $\mathcal{N}(0,1)$.
  \item Para cada tamaño $n\in\{2,3,4,5,6,7,8\}$ se repiten $N=10$
        ejecuciones y se promedia el tiempo.
  \item El tiempo se mide con \verb|time.perf_counter()|.
\end{itemize}

\subsubsection{Código Python}
\begin{verbatim}
import numpy as np, time
from commuting_basis import commuting_basis

sizes = range(2, 9)          # n = 2 … 8
trials = 10
results = []
for n in sizes:
    total = 0.0
    for _ in range(trials):
        A = np.random.randn(n, n)
        t0 = time.perf_counter()
        commuting_basis(A)
        total += time.perf_counter() - t0
    results.append(total / trials)
print("n  time (s)")
for n, t in zip(sizes, results):
    print(f"{n:2d} {t:8.5f}")
\end{verbatim}

\subsubsection{Resultados típicos}
\begin{center}
\begin{tabular}{cc}
\toprule
$n$ & Tiempo medio (s) \\
\midrule
2 & 0.0003 \\
3 & 0.0011 \\
4 & 0.0048 \\
5 & 0.0205 \\
6 & 0.0932 \\
7 & 0.3801 \\
8 & 1.6900 \\
\bottomrule
\end{tabular}
\end{center}

\subsubsection{Análisis}
El crecimiento es aproximadamente \emph{cúbico–cuártico} en $n$:
la dimensión del problema SVD es $n^2\times n^2$; el costo nominal de la SVD
completa es $\mathcal O(n^6)$.  La curva empírica confirma una explosión
rápida del tiempo al superar $n\approx 8$ en una CPU estándar.

\subsection{Optimización}
El código completo del algoritmo optimizado para matrices triangulares (\texttt{matriz\_triangular.py}) se encuentra en el repositorio de GitHub:

\begin{center}
\url{https://github.com/gustavop-dev/taller_1_algebra_lineal_aplicada_unal/blob/master/conmutatividad/matriz_triangular.py}
\end{center}


\end{document}